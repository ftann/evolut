%
% Perspective
%

% !TEX root = ../../main.tex

\chapter{Diskussion und Ausblick}

  \todo[inline]{Bespricht die erzielten Ergebnisse bezüglich ihrer Erwartbarkeit, Aussagekraft und Relevanz}
  \todo[inline]{Interpretation und Validierung der Resultate}
  \todo[inline]{Rückblick auf Aufgabenstellung, erreicht bzw\@. nicht erreicht}
  \todo[inline]{Legt dar, wie an die Resultate (konkret vom Industriepartner oder weiteren
  Forschungsarbeiten; allgemein) angeschlossen werden kann; legt dar, welche Chancen die
  Resultate bieten}
  \section{Diskussion}
    Die gezeigte Problemstellung ist keinesfalls trivial. Es stellt sich heraus, das eine Simulation jeweils viel Zeit beansprucht
    und äusserst rechenintensiv ist.
    Die erstelle Simulationsapplikation ist gut erweiterbar und ermöglicht Interessierten sie weiter auszubauen.
    Die fehlende Implementation der Reaktion auf das Feedback der Steuerung ist sicher die Komponente,
    welche am meisten helfen würde, bessere Resultate zu finden.
    Ein weiterer Punkt ist die langsame und teilweise fehlerhafte Physik-Engine p2.js.
    Das Ersetzen der Physik-Engine wäre der zweitwichtigste Schritt um bessere Resultate zu finden.
    Mit einer besseren Physik-Engine könnte auch schneller simuliert werden.
    Für einen 30 Sekunden langen Simulationslauf, benötigt die Engine zum Teil 1 Minute.
    Es kann auch die Hypothese aufgestellt werden, dass JavaScript V8 momentan noch nicht ausreichend schnell genug ist.
    Ebenfalls ein spanenndes Thema, jedoch wäre das eher etwas für eine reine IT Bachelor-Arbeit.
    Leider erlaubte es die Zeit nicht die Hypothese mit den Selektionsstrategien zu validieren.
    Diese Hyptothese würd auch genügend Stoff für eine weitere Arbeit liefern.
    Der Parcour wird momentan nur mit einem Typ von Element generiert, ein Höhenfeld das einen Berg repräsentiert.
    Anderes Terrain wie Eis, Wasser oder Grass könnte auch noch implementiert werden.
    Um eine Aussage über die Hyptothese der Körperpunkte \ref{subsub:hypoKp} zu treffen,
    müsste wie erwähnt unter \ref{subsub:bpScnd} die Mutation der Anzahl implementiert werden.
