%
% Perspective
%

% !TEX root = ../../main.tex

\chapter{Diskussion und Ausblick}

  \todo[inline]{Bespricht die erzielten Ergebnisse bezüglich ihrer Erwartbarkeit, Aussagekraft und Relevanz}
  \todo[inline]{Interpretation und Validierung der Resultate}
  \todo[inline]{Rückblick auf Aufgabenstellung, erreicht bzw\@. nicht erreicht}
  \todo[inline]{Legt dar, wie an die Resultate (konkret vom Industriepartner oder weiteren
  Forschungsarbeiten; allgemein) angeschlossen werden kann; legt dar, welche Chancen die
  Resultate bieten}



%\subsubsection{Hypothese Selektionsstrategie\label{sub:Hypothese Selektionsstrategie}}

  %Es wird die Hypothese aufgestellt,
  %dass turnierbasierte Selektion die besseren Resultate liefert im Vergleich zu den folgenden Selektionsstrategien: Rangbasierte Selektion und proportionale Selektion.
  %Um dies zu validieren ist es erforderlich die Selektionsstrategien rangbasierte Selektion und proportionale Selektion zu implementieren.
  %Ob dies zutrifft wird in Kapitel~\ref{chap:Diskussion} diskutiert.

  \section{Simulation}

    Es stellt sich heraus, das eine Simulation viel Zeit beansprucht und äusserst rechenintensiv ist.
    Die erstelle Simulationsapplikation ist gut erweiterbar und ermöglicht Interessierten sie weiter auszubauen.
    \\
    Eine Verbesserungsmöglichkeit ist die langsame und teilweise fehlerhafte Physik-Engine p2.js auszutauschen.
    Mit Hilfe einer anderen Physik-Engine kann schneller simuliert werden und es können bessere Resultate gefunen werden.
    Für einen 30 Sekunden langen Simulationslauf, benötigt die Engine zum Teil eine Minute.
    \\
    Es kann die Hypothese aufgestellt werden, dass die JavaScript-Engine V8 momentan noch nicht ausreichend schnell ist.
    Eine weiterführende Arbeit kann untersuchen wie die Leistungen anderer Implementationen von Javascript-Engines im
    Vergleich zu V8 abschneiden.
    \\
    Die zeitliche Beschränkung dieser Arbeit hat es nicht erlaubt
    die Hypothese mit verschiedenen Selektionsstrategien weiter auszuführen~\vref{sub:Hypothese Selektionsstrategie}.
    Leider erlaubte es die Zeit nicht die Hypothese mit den Selektionsstrategien zu validieren.
    Diese Hyptothese würd auch genügend Stoff für eine weitere Arbeit liefern.
    \\
    Der Parcours wird nur mit einem Typ von Element generiert, ein Höhenfeld welches einen Berg repräsentiert.
    Neue Terrain-Typen wie Eis, Wasser oder Grass könnten zusätzlich hinzugefügt werden.
    \\
    Um eine Aussage über die Hyptothese der Körperpunkte~\vref{subsub:hypoKp} zu tvreffen,
    müsste wie erwähnt unter~\vref{subsub:bpScnd} die Mutation der Anzahl Körperpunkte implementiert werden.

  \section{Feedback an den Bewegungsmotor}

    Die fehlende Implementation der Reaktion auf das Feedback der Steuerung ist sicher die Komponente,
    welche am meisten helfen würde, bessere Resultate zu finden.

  \section{Resultate}

    Die gefundenen Resultate wiederspiegeln Lösungen innerhalb einer künstlichen Umgebung.
    In dieser Umgebung wird eine vereinfachte Physik verwendet, welche nur an die reale Physik angenähert ist.
    Deshalb sind die Resultate nicht auf die reale Welt übertragbar.

    % Tendenz?
