%
% Method
%

% !TEX root = ../../main.tex

\chapter{Methode}
  \todo[inline]{Beschreibt die Grundüberlegungen der realisierten Lösung (Konstruktion/Entwurf) und die
  Realisierung als Simulation, als Prototyp oder als Software-Komponente}
  \todo[inline]{Definiert Messgrössen, beschreibt Mess- oder Versuchsaufbau, beschreibt und dokumentiert
  Durchführung der Messungen/Versuche}
  \todo[inline]{Experimente}
  \todo[inline]{Lösungsweg}
  \todo[inline]{Modell}
  \todo[inline]{Tests und Validierung}
  \todo[inline]{Theoretische Herleitung der Lösung}

  \section{Design der Individuen}

    \subsection{Koerper}
    \label{sub:Koerper}
      Die erste Frage wechlse sich stellt ist, wie kann der Koerper am besten beschrieben werden? \\
      Der Koeper setzt sich aus verschiedenen Punkten auseinander. Die mindest Anzahl solcher Punkte wird auf 4 in einem ersten Schritt festgelegt. \\
      Die Obergrenze bei 8. Dies erlaubt dem Koeper eine polygonartige Figur.
    \subsection {Beine}
    \label{sub:Beine}
      Insgesammt weissen die Individuuen 6 Beine auf. Ein Beinpaar ist symmetrisch. Ein Bein hat jeweils 2 Gelenke, eines welches es mit dem Koerper verbindet,
      das andere welches Ober- mit Unterschenkel verbindet. Als Vereinfachung wird die Hoehe des Gelenkes 0 gesetzt  \(h_{a} = 0\)
      Das Bein weisst eine Hoehe \(h_{l}\) auf. Die Position des Gelenkes,welches die beiden Beinschenkel verbindet,
      wird als Abstand zum oberen Ende des Oberschenkel definiert \(h_{t}\), andersrum gesagt ist das die Hoehe des Oberschenkels.
      Die Hoehe des Unterschenkels wird als \(h_{s}\) bezeichnet.
      Ein Bein weisst eine Breite auf \(w_{l}\). Am Ende des Unterschenkels wird ein Stumpf angehaengt.
    \subsubsection{Masse}
    \label{subs:Masse}
      Das ganze Individum weisst eine fixe Masse von 1 auf \(m = 0\). Die Masse wird aber auf die einzelnen Koerpeteile verteilt.
      \\
      %
% Concept leg
%

% !TEX root = ../main.tex

\begin{tikzpicture}[scale=0.4]

  %def
  \tkzInit[xmin=-15,xmax=12,ymin=-6,ymax=12]
  \tkzDrawXY[noticks]
  \tkzDefPoints{-6/4/C1, 2/1/C2,0.205/6.548/A,-3/-2/B}

  %drawing
  \tkzDrawPoints(A,B)
  \tkzDrawSegments[gray](C1,C2)
  \tkzDrawSegments[red](C2,A C2,B)
  \tkzDrawSegments[blue](C1,A C1,B)
  \tkzDrawCircle(C1,A)
  \tkzDrawCircle(C2,A)
  % notation
  \tkzLabelPoints(C2)
  \tkzLabelPoints[left](C1)
  \tkzLabelPoints[above](A)
  \tkzLabelPoints[below](B)
  \tkzMarkSegment[color=blue,pos=.5,mark=||](C1,B)
  \tkzMarkSegment[color=red,pos=.5,mark=|](B,C2)
  \tkzMarkSegment[color=blue,pos=.5,mark=||](C1,A)
  \tkzMarkSegment[color=red,pos=.5,mark=|](A,C2)
  \tkzMarkRightAngle[size=0.7](C1,A,C2)
  \tkzMarkRightAngle[size=0.7](C1,B,C2)
  \tkzLabelCircle[above left](C1,A)(180){$K_1$}
  \tkzLabelCircle[right](C2,A)(-60){$K_2$}

\end{tikzpicture}

\begin{tikzpicture}[scale=0.4]

  \tkzInit[xmin=-1,xmax=11,ymin=-1,ymax=21]
  \tkzClip

  \tkzDefPoint(0,0){O}
  \tkzDefPoint(7,0){A}
  \tkzDefPoint(0,20){H}
  \tkzDefPoint(0,6){Hj}
  \tkzDefPoint(0,1){Hf}
  \tkzDefPoint(7,1){Hfm}
  \tkzDefPoint(7,20){J1}
  \tkzDefPoint(7,11){J2}
  \tkzDefPoint(10,20){N}

  \tkzDrawPoints(J1,J2)
  \tkzDefSquare(O,N) \tkzGetPoints{C}{D}
  \tkzDrawSegments[gray](A,J2)
  \tkzDrawSegments[gray](J2,J1)

  % Foot
  \tkzDrawArc[angles](Hfm,A)(180,360) \tkzGetPoints{Hfr}{Hfl}
  \tkzDrawSegments(Hfr,Hfl)
  \tkzLabelPoints(Hfm)

\end{tikzpicture}

