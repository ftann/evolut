%
% Method
%

% !TEX root = ../../main.tex

\chapter{Methode}
  \todo[inline]{Beschreibt die Grundüberlegungen der realisierten Lösung (Konstruktion/Entwurf) und die
  Realisierung als Simulation, als Prototyp oder als Software-Komponente}
  \todo[inline]{Definiert Messgrössen, beschreibt Mess- oder Versuchsaufbau, beschreibt und dokumentiert
  Durchführung der Messungen/Versuche}
  \todo[inline]{Experimente}
  \todo[inline]{Lösungsweg}
  \todo[inline]{Modell}
  \todo[inline]{Tests und Validierung}
  \todo[inline]{Theoretische Herleitung der Lösung}

  \section{Design der Individuen}

    \subsection{Körper}
    \label{sub:Körper}
      Die erste Frage wechlse sich stellt ist, wie kann der Koerper am besten beschrieben werden? \\
      Der Koeper setzt sich aus verschiedenen Punkten auseinander. Die mindest Anzahl solcher Punkte wird auf 4 in einem ersten Schritt festgelegt. \\
      Die Obergrenze bei 8. Dies erlaubt dem Koeper eine polygonartige Figur.
      \subsubsection{Hyptohese Körperpunkte}
      \label{subsub:hypoKp}
      Es wird die Hypothese aufgestellt, dass sich ein Indiviuum mit 8 Körperpunkten schneller durch den Parcour bewegen kann, als ein Individuum mit weniger Körperpunkten.
      Die Vermutung dabei ist, dass durch die komplexere Form des Indiviuum, es besser die Balance halten kann.
      Ob dies zutrifft wird in Kapitel \ref{chap:Resulate} diskutiert.
    \subsection {Beine}
    \label{sub:Beine}
      Insgesammt weissen die Individuuen 6 Beine auf. Ein Beinpaar ist symmetrisch. Ein Bein hat jeweils 2 Gelenke, eines welches es mit dem Koerper verbindet,
      das andere welches Ober- mit Unterschenkel verbindet. Als Vereinfachung wird die Hoehe des Gelenkes 0 gesetzt  \(h_{a} = 0\)
      Das Bein weisst eine Hoehe \(h_{l}\) auf. Die Position des Gelenkes,welches die beiden Beinschenkel verbindet,
      wird als Abstand zum oberen Ende des Oberschenkel definiert \(h_{t}\), andersrum gesagt ist das die Hoehe des Oberschenkels.
      Die Hoehe des Unterschenkels wird als \(h_{s}\) bezeichnet.
      Ein Bein weisst eine Breite auf \(w_{l}\). Am Ende des Unterschenkels wird ein Stumpf angehaengt.
    \subsubsection{Masse}
    \label{subs:Masse}
      Das ganze Individum weisst eine fixe Masse von 1 auf \(m = 0\). Die Masse wird aber auf die einzelnen Koerpeteile verteilt.
      \\
      %
% Concept leg
%

% !TEX root = ../main.tex

\begin{tikzpicture}[scale=0.4]

  %def
  \tkzInit[xmin=-15,xmax=12,ymin=-6,ymax=12]
  \tkzDrawXY[noticks]
  \tkzDefPoints{-6/4/C1, 2/1/C2,0.205/6.548/A,-3/-2/B}

  %drawing
  \tkzDrawPoints(A,B)
  \tkzDrawSegments[gray](C1,C2)
  \tkzDrawSegments[red](C2,A C2,B)
  \tkzDrawSegments[blue](C1,A C1,B)
  \tkzDrawCircle(C1,A)
  \tkzDrawCircle(C2,A)
  % notation
  \tkzLabelPoints(C2)
  \tkzLabelPoints[left](C1)
  \tkzLabelPoints[above](A)
  \tkzLabelPoints[below](B)
  \tkzMarkSegment[color=blue,pos=.5,mark=||](C1,B)
  \tkzMarkSegment[color=red,pos=.5,mark=|](B,C2)
  \tkzMarkSegment[color=blue,pos=.5,mark=||](C1,A)
  \tkzMarkSegment[color=red,pos=.5,mark=|](A,C2)
  \tkzMarkRightAngle[size=0.7](C1,A,C2)
  \tkzMarkRightAngle[size=0.7](C1,B,C2)
  \tkzLabelCircle[above left](C1,A)(180){$K_1$}
  \tkzLabelCircle[right](C2,A)(-60){$K_2$}

\end{tikzpicture}

\begin{tikzpicture}[scale=0.4]

  \tkzInit[xmin=-1,xmax=11,ymin=-1,ymax=21]
  \tkzClip

  \tkzDefPoint(0,0){O}
  \tkzDefPoint(7,0){A}
  \tkzDefPoint(0,20){H}
  \tkzDefPoint(0,6){Hj}
  \tkzDefPoint(0,1){Hf}
  \tkzDefPoint(7,1){Hfm}
  \tkzDefPoint(7,20){J1}
  \tkzDefPoint(7,11){J2}
  \tkzDefPoint(10,20){N}

  \tkzDrawPoints(J1,J2)
  \tkzDefSquare(O,N) \tkzGetPoints{C}{D}
  \tkzDrawSegments[gray](A,J2)
  \tkzDrawSegments[gray](J2,J1)

  % Foot
  \tkzDrawArc[angles](Hfm,A)(180,360) \tkzGetPoints{Hfr}{Hfl}
  \tkzDrawSegments(Hfr,Hfl)
  \tkzLabelPoints(Hfm)

\end{tikzpicture}


  \section{Auswahl des Evolutinären Algorithmus}
    Es stehen 4 Typen von Evolutionäre Algorithmen zur Auswahl:
    \begin{itemize}
      \item Genetische Programmierung \ref{item:genProg}
      \item Genetischer Algorithmus \ref{item:genAlgo}
      \item Evolutiäre Strategie \ref{item:evStart}
      \item Evolutinäre Programmierung \ref{item:evProg}
    \end{itemize}
    \todo[inline]{Genauer Ausführen warum}
    Um die Problemstellung zu lösen, wurde entschieden Evolutinäre Programmierung einzusetzten.
    Warum Evolutinäre Programmierung eingesetzt wird, wird nachfolgend erläutert. Das Austauschen der Genen, Rekombination, macht keinen Sinn wenn man Individuen das Laufen beibringt.
    Man stelle sich dazu vor, man tausche das Gen welches den Körper definiert, mit dem der Beine. Dies stellt schon den ersten Indikator dafür da,
    dass Evolutionäre Programmierung eingesetzt werden kann. Ein weiteres Kriterium ist die genetische Repräsentation.
    Da unser Indiviuum sich aus Punkten im einem Koordinatensystem, verschiedene Winkel und einem Antrieb(Motor) definiert, wird die reale Werte Repräsentation eingesetzt.
    Eine Baumrepräsenation oder binäre Repräsentation erscheint hier wenig gewinnbringend. Ein Punkt ist nichts anderes als zwei reale Werte, ein Winkel kann ebenfalls so beschrieben werden und ein Antrieb auch.
    Die Problemstellung würde wahrscheinlich anderst bewältigt werden, würde es hier um reale Kreaturen gehen und nicht nur um virtuelle Roboter, welche in einer künstlichen Umgebung(virtuelles Koordinatensystem) das Laufen lernen.

  \section{Eingesetzte Technologien}
    Zur Auswahl standen mehrere Programmiersprachen und Physik Engines zur Auswahl. Die Evaluationskriterien dabei waren:
    \begin{itemize}
      \item Platform Interoperabilität (Linux, Windows Mac OS X)
      \item Einfache Handhabung
      \item Stablität
      \item Features der Physik Engine
      \item Ökosystem der Programmiersprache (verfügbare Bibilotheken)
    \end{itemize}

     Als erstes wurde F\# und diverse populäre .NET Physics Engines wie Farseer Physics, Physics 2D und Digital Rune Engine evaluiert.
     F\# ist eine neue funktionale Sprache die mit einer sauberen Synatx und einem der besten Typensystem überhaupt besticht.
     Jedoch gestaltete sich die Konfiguration und Platform Interoperablität sehr schwierig. Dies lag nich an den Physic Engines selber,
     sondern an der Open Source Implementiertung Mono Game. Eine plattformübergreifende Konfiguration,
     damit mit allen oben genannten Betriebssystemen entwickelt werden kann, konnte nicht erstellt werden.\\
     Als zweite Möglichkeit wurde JavaScript und die Physics Engines Matter.js und p2.js in Betracht gezogen.
     Die Popularität von JavaScript hat in den letzten Jahren enorm zugenommen.
     Auch wurde mit dem neuen ECMA6 Standard die Sprache enorm verbessert und viele neue Features eingeführt. Das JavaScript Ökosystem ist eines der grössten überhaupt.
     Da JavaScript eine Webtechnologie ursprünglich ist und auf allen Plattformen problemlos läuft, wurde das Kriterium der Platform Interoperabilität ohne zusätzliche Konfiguration erfüllt.
     Matter.js wirkte noch nicht so ausgereift wie p2.js und wies viele Fehler auf. Um JavaScript und p2.js ohne Browser zu betreiben wird Electron eingesetzt.
     Electron ist nicht anderes als Node.js und Google Chrome vereint in einer Standalone Anwendung.
     Mit Hilfe von Node.js lassen sich alle I/0 Zugriffe regeln und es kann auf den Node Package Manager zur Abhänigskeitsverwaltung von anderen Bibliotheken zurück gegriffen werden.
     Auch lässt sich unter bestimmten Umständen JavaScript schneller ausführen unter Node.js als in einem Browser.
