%
% Theory
%

% !TEX root = ../../main.tex

\chapter{Grundlagen}

\section{Natural Evoltions vs Artifical}
  Natürliche Evolution hat kein vordefiniertes Ziel und ist ein sogennanter open-ended Anpassungsprozess. Artifizielle Evolution jedoch ist ein Optimierungsprozess, welcher versucht Lösungen zu vordefinierten Problem zu finden \cite[S.1]{book:bioInspired}. \\
  \todo[inline]{intelligent design}

\section{Artifical Evolution}

  \subsection{Evolutionärer Algorithmus erstellen}
  \label{sub:evAlgoErstellen}
    Nach \cite[S.16 - 29]{book:bioInspired}:
    \begin{itemize}

      \item Auswahl einer genetischen Repräsentation
        \begin{itemize}
          \item Diskrete Repräsentation [ABCDEF], [0111110]
          \item Reale Werte Pepräsentation 12.1245
          \item Baum Repräsentation \Tree [.A [.B [.C eins ] [.D zwei ] ].B [.E {3 und 4} ] ].A
        \end{itemize}

      \item Population erstellen
          \begin{itemize}
            \item Die Grösse der Population hängt von den Eigenschaften des Suchraums und der Rechenkosten der Evaluation ab
            \item Möglich viele unterschiedliche Individuen generieren
          \end{itemize}

      \item Fitnessfunktion definieren
        \begin{itemize}
          \item Auswahl und Kombination der Fitnesskomponenten
          \item Wie wird die Funktion evaluiert?
        \end{itemize}

      \item Auswahloperation definieren
        \begin{itemize}
          \item Ziel: Möglichst viele gute Individuen selektieren um Nachkommen der nächsten Generation zu erzeugen
          \item Selection Pressure
            \begin{itemize}
              \item Prozentsatz der Population die verwendet wird um Nachkommen zu reproduzieren, bsp 0.2
              \item Vorteil: Schnelle Verbesserung der Fitnessfunktion
              \item Nachteil: Schwierig Diversität zu erhalten
            \end{itemize}
          \item Proportinate Selection
            \begin{itemize}
              \item Reproduktionsrate proportinal zur Fitnessfunktion
              \item Nachteil: Schlecht wenn alle Indivduen gleiche Fitnesswerte vorweisen oder es nur einen Ausreisser gibt
            \end{itemize}
          \item Rank-based Selection
            \begin{itemize}
              \item Individuen Rangliste erstellen und Reproduktionswahrscheinlichkeiten proportional zu Rang zu ordnen
              \item Vorteil: Da die Reproduktionswahrscheinlichkeit proportinal zum Rang ist, kommt es nicht drauf an wie wenig unterschiedlich die Individuen sind.
            \end{itemize}
          \item Truncated Rank-based Selection: Nur die besten Individuen der Rangliste zur Reproduktion selektieren. Im Gegensatz zu Rank-based Selection erzeugt jedes Individuum die gleiche Anzahl Nachkommen.
          \item Tournament-basd Selection
            \begin{itemize}
              \item Wähle k zufällige Individuen, das Individuum mit der höchsten Fitness erzeugt Nachkommen
              \item Vorteil: Gute Balance zwischen Selection-Pressure und genetischer Diversität
            \end{itemize}
          \item Generational Replacement
            \begin{itemize}
              \item Die ganze Population wird durch Nachkommen ersetzt
              \item Elitism kann helfen den Prozess zu verbessern in dem immer die n besten Individuen erhalten bleiben
            \end{itemize}


        \end{itemize}

      \item Rekombinationsfunktion definieren
        \begin{itemize}
          \item Paarweise seletkion der Nachkommen
          \item Gene der Paare werden rekombiniert (untereinander vertauscht)
          \item One-Point Crossover
            \begin{itemize}
              \item Zufällige Bestimmung eines Cross-Over Punktes an dem die Gene des Paares vertauscht werden.
              \item Anwendbar auf Diskrete und Reale Werte Repräsentationen
            \end{itemize}
          \item Multi-Point Crossover: Gleich wie One-Points, es werden aber mehrere Cross-Over Punkte bestimmt
          \item Uniform Crossover
            \begin{itemize}
              \item Vertauschen von Genen an n zufälligen Positionen
              \item Anwendbar auf Reale Werte Repräsentationen
            \end{itemize}
          \item Arithemtic Crossover
            \begin{itemize}
              \item Durchschnitte von Genen an n zufälligen Positionen
              \item Es wird nur ein Nachkomme generiert
              \item Anwendbar auf Reale Werte Repräsentationen
            \end{itemize}
          \item Sequenzen : Es wird ein neuer Nachkomme gebildet unter Einhalten von Regeln. Alle Einträge dürfen nur Einmal vorkommen etc.... Unique Version von Multi-Point Crossover.
          \item Rekombinations bei Baum Repräsentationen: Es werden nur zufällige Teile des Baumes unter den Paaren vertauscht.
        \end{itemize}

      \item Mutationsfunktion definieren
        \begin{itemize}
          \item Mutationen operieren auf Level Indivduum
          \item Es ist Vorsicht geboten, da gefundene Lösungen durch Mutationen verloren gehen können
          \item Positionen eine Genomes werden mit einer bestimmten Wahrscheinlichkeit \(p_{m}\) mutiert.
          \item Binär: Flippen der Bits
          \item Reale Werte Repräsentation: Addieren eines zufälligen Wertes aus einer Gaus Verteilung \(N(0,\sigma)\)
          \item Sequenzen: Zufälliges austauschen Zweier Positionen
          \item Baum: Mutieren von Nodes. Gleiches Alphabet verwenden.
        \end{itemize}

      \item Ergebniss Analysemethode definieren
    \end{itemize}

    Genetische Representation: Erbgutdaten, genetic encoding \\
    Phenotyp: Aussehen des Individum -> Manifestation. In unserem Fall die 6-Beinigen Kreaturen als Pixelgrafik. \\
    Initialisieren -> Simulieren -> Selection -> Rekombination -> Mutation -> Analyse \\
    Simulieren bis Analayse solange wiederholen bis zufrieden mit Resultat


  \subsection{Arten von evolutionären Algorithmen}
  \label{sub:artenEvAlgos}
    \begin{itemize}
      \item Genetische Algorithmen
      \label{item:genAlgo}
        \begin{itemize}
          \item Arbeiten mit binären Represäntationen
          \item Setzen Cross-Over und Mutationen ein
        \end{itemize}
      \item Genetische Programmierung
      \label{item:genProg}
        \begin{itemize}
          \item Arbeitet mit Bäumen
          \item Setzt Cross-Over und Mutationen ein
        \end{itemize}
      \item Evolutionäre Programmierung
      \label{item:evProg}
        \begin{itemize}
          \item Arbeitet mit Realen Werte Repräsentationen
          \item Setzt nur Mutationen ein
          \item Oft wird Tournament-based Selection eingesetzt
        \end{itemize}
      \item Evolutionäre Strategien
      \item{item:evStrat}
        \begin{itemize}
          \item Gleich wie \ref{item:evProg} aber die Varianz der Verteilung die für Mutationen gebraucht werden ist genetische codiert.
        \end{itemize}
    \end{itemize}
