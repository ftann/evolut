%
% Theory
%

% !TEX root = ../../main.tex

\chapter{Grundlagen}

In diesem Kapitel werden die Grundlagen zu Evolutionären Algorithmen aufgearbeitet.

\section{Natural Evoltions vs Artifical}
  Natürliche Evolution hat kein vordefiniertes Ziel und ist ein sogennanter open-ended Anpassungsprozess. Artifizielle Evolution jedoch ist ein Optimierungsprozess, welcher versucht Lösungen zu vordefinierten Problem zu finden \cite[S.1]{book:bioInspired}. \\
  \todo[inline]{intelligent design}

\section{Artifical Evolution}

  \subsection{Schritte eines Evolutinären Algorithmus}
  \label{sub:stepsEvAlgo}
    Nach \cite[S.16 - 29]{book:bioInspired}: \\

    \subsubsection{Genetische Repräsenation}
    Der erste Schritt bei der Definition eines Evolutionären Alogrithmus ist die Auswahl der genetischen Repräsentation.
    Nicht alle Arten von Evolutionären Algorithmen harmonieren mit jeder Repräsentation.

      \paragraph{Diskrete Repräsentation}
      \label{par:diskreteRepräsentation}
        Bei der diskreten Repräsentation wird das Genom als binären String dargestellt [0111110].
        Dieser binäre String kann anschliessend zu einem Phenotyp übersetzt werden.
        Zum Beispiel kann eine Bitsequenz, direkt zu einer Zahl als Phenotyp übersetzt werden [0011] -> 3.
        Ebenfalls kann eine diskrete Repäsentation eine Folge von beliebigen Zeichen annehmen [ABCDEF].
        D. Floreano und C. Mattiussi bringen dazu ein gutes Beispiel anhand des Travelling Sales Man Problemes an \cite[S.18]{book:bioInspired}:
        Jeder Buchstabe in der Sequenz repräsentiert dabei einen Ort, welchen es zu besuchen gilt.

      \paragraph{Reale Werte Repräsentation}
      \label{par:Reale Werte Repräsentation}
        Weiter kann die Reale Werte Repräsentation gewählt werden. Das Genom wird hierbei durch reele Zahlen repräsentiert.
        Beispielsweisse kann die optimale Position eines Zimmers (beste Flächennutzung) in einem Haus durch reale Zahlen dargestellt werden.

      \paragraph{Baum Repräsentationen}
      \label{par:Reale Werte Repräsentation}
        Das zu evolvierende Objekt kann durch einen Baum dargestellt werden. \Tree [.A [.B [.C eins ] [.D zwei ] ].B [.E {3 und 4} ] ].A
        Baum Repräsenationen werden vorallem bei der genetischen Programmierung eingesetzt.

    \subsubsection{Initiale Population}
      Die Grösse der Population hängt von den Eigenschaften des Suchraums und der Rechenkosten der Evaluation ab.
      Wichtig bei der Erstellung einer Population dabei ist, möglichst diverse Individuen zu erstellen,
      damit man sich nicht schon von Anfang an viele mögliche Lösung verspielt.

    \subsubsection{Fitness Funktion}
      Mit Hilfe der Fintess Funktion lassen sich Individuen beurteilen, wie gut geeiget Ihre Gene sind um die Problemstellung zu bewältigen.
      In der natürlichen Evolution ist die Fitness des Tieres, wie viele Nachkommen es erzeugen kann. In der technischen Welt jedoch muss der
      Anwenden sie jeweils selber definieren und nach seiner Problemstellung anpassen.

    \subsubsection{Selektionsoperation}
      Eine Selektionsoperation hilft einem möglichst gutgeignete Indivduen einer Generation zu selektieren.
      Die Selektierten bilden die Basis für die nächste Generation. Eine grosse Herausforderung beim Selektieren ist die Erhaltung der Diversität.

      \paragraph{Selection Pressure}
        Als Selection Pressure wird der Prozensatz der Individuen der aktuellen Generation, welche man verwendet um Nachkommen zu erzeugen, bezeichnet.
        Dabei werden am nur die fittesten Individuen selektiert. Zum Beispiel bei 10 Individuen werden nur die 4 selektiert, welche den höchsten Fitnesswert aufweisen.
        Selektion nach Selection Pressure hat den Nachteil, dass die Diversität schnell verloren geht.

    \begin{itemize}

      \item Auswahloperation definieren
        \begin{itemize}
          \item Proportinate Selection
            \begin{itemize}
              \item Reproduktionsrate proportinal zur Fitnessfunktion
              \item Nachteil: Schlecht wenn alle Indivduen gleiche Fitnesswerte vorweisen oder es nur einen Ausreisser gibt
            \end{itemize}
          \item Rank-based Selection
            \begin{itemize}
              \item Individuen Rangliste erstellen und Reproduktionswahrscheinlichkeiten proportional zu Rang zu ordnen
              \item Vorteil: Da die Reproduktionswahrscheinlichkeit proportinal zum Rang ist, kommt es nicht drauf an wie wenig unterschiedlich die Individuen sind.
            \end{itemize}
          \item Truncated Rank-based Selection: Nur die besten Individuen der Rangliste zur Reproduktion selektieren. Im Gegensatz zu Rank-based Selection erzeugt jedes Individuum die gleiche Anzahl Nachkommen.
          \item Tournament-basd Selection
            \begin{itemize}
              \item Wähle k zufällige Individuen, das Individuum mit der höchsten Fitness erzeugt Nachkommen
              \item Vorteil: Gute Balance zwischen Selection-Pressure und genetischer Diversität
            \end{itemize}
          \item Generational Replacement
            \begin{itemize}
              \item Die ganze Population wird durch Nachkommen ersetzt
              \item Elitism kann helfen den Prozess zu verbessern in dem immer die n besten Individuen erhalten bleiben
            \end{itemize}


        \end{itemize}

      \item Rekombinationsfunktion definieren
        \begin{itemize}
          \item Paarweise seletkion der Nachkommen
          \item Gene der Paare werden rekombiniert (untereinander vertauscht)
          \item One-Point Crossover
            \begin{itemize}
              \item Zufällige Bestimmung eines Cross-Over Punktes an dem die Gene des Paares vertauscht werden.
              \item Anwendbar auf Diskrete und Reale Werte Repräsentationen
            \end{itemize}
          \item Multi-Point Crossover: Gleich wie One-Points, es werden aber mehrere Cross-Over Punkte bestimmt
          \item Uniform Crossover
            \begin{itemize}
              \item Vertauschen von Genen an n zufälligen Positionen
              \item Anwendbar auf Reale Werte Repräsentationen
            \end{itemize}
          \item Arithemtic Crossover
            \begin{itemize}
              \item Durchschnitte von Genen an n zufälligen Positionen
              \item Es wird nur ein Nachkomme generiert
              \item Anwendbar auf Reale Werte Repräsentationen
            \end{itemize}
          \item Sequenzen : Es wird ein neuer Nachkomme gebildet unter Einhalten von Regeln. Alle Einträge dürfen nur Einmal vorkommen etc.... Unique Version von Multi-Point Crossover.
          \item Rekombinations bei Baum Repräsentationen: Es werden nur zufällige Teile des Baumes unter den Paaren vertauscht.
        \end{itemize}

      \item Mutationsfunktion definieren
        \begin{itemize}
          \item Mutationen operieren auf Level Indivduum
          \item Es ist Vorsicht geboten, da gefundene Lösungen durch Mutationen verloren gehen können
          \item Positionen eine Genomes werden mit einer bestimmten Wahrscheinlichkeit \(p_{m}\) mutiert.
          \item Binär: Flippen der Bits
          \item Reale Werte Repräsentation: Addieren eines zufälligen Wertes aus einer Gaus Verteilung \(N(0,\sigma)\)
          \item Sequenzen: Zufälliges austauschen Zweier Positionen
          \item Baum: Mutieren von Nodes. Gleiches Alphabet verwenden.
        \end{itemize}

      \item Ergebniss Analysemethode definieren
    \end{itemize}

    Genetische Representation: Erbgutdaten, genetic encoding \\
    Phenotyp: Aussehen des Individum -> Manifestation. In unserem Fall die 6-Beinigen Kreaturen als Pixelgrafik. \\
    Initialisieren -> Simulieren -> Selection -> Rekombination -> Mutation -> Analyse \\
    Simulieren bis Analayse solange wiederholen bis zufrieden mit Resultat


  \subsection{Arten von evolutionären Algorithmen}
  \label{sub:artenEvAlgos}
    \begin{itemize}
      \item Genetische Algorithmen
      \label{item:genAlgo}
        \begin{itemize}
          \item Arbeiten mit binären Represäntationen
          \item Setzen Cross-Over und Mutationen ein
        \end{itemize}
      \item Genetische Programmierung
      \label{item:genProg}
        \begin{itemize}
          \item Arbeitet mit Bäumen
          \item Setzt Cross-Over und Mutationen ein
        \end{itemize}
      \item Evolutionäre Programmierung
      \label{item:evProg}
        \begin{itemize}
          \item Arbeitet mit Realen Werte Repräsentationen
          \item Setzt nur Mutationen ein
          \item Oft wird Tournament-based Selection eingesetzt
        \end{itemize}
      \item Evolutionäre Strategien
      \item{item:evStrat}
        \begin{itemize}
          \item Gleich wie \ref{item:evProg} aber die Varianz der Verteilung die für Mutationen gebraucht werden ist genetische codiert.
        \end{itemize}
    \end{itemize}
