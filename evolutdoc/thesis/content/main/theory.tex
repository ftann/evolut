%
% Theory
%

% !TEX root = ../../main.tex

\chapter{Grundlagen}

In diesem Kapitel werden die Grundlagen zu Evolutionären Algorithmen aufgearbeitet.

\section{Natural Evoltions vs Artifical}

  Natürliche Evolution hat kein vordefiniertes Ziel und ist ein sogennanter open-ended Anpassungsprozess.
  Artifizielle Evolution jedoch ist ein Optimierungsprozess,
  welcher versucht Lösungen zu vordefinierten Problem zu finden \cite[S.1]{book:bioInspired}.
  \\
  Das Ziel ist ein optimal angepasstes Individuum zu finden, weshalb artifizeille Evolution verwendet wird.

  \subsection{Intelligent Design}
  \label{sub:IntelligentDesign}

    \todo[inline]{intelligent design}

\section{Artifical Evolution}

  \subsection{Schritte eines Evolutinären Algorithmus}
  \label{sub:stepsEvAlgo}
    Nach \cite[S.16 - 29]{book:bioInspired}: \\

    \subsubsection{Genetische Repräsenation}
    Der erste Schritt bei der Definition eines Evolutionären Alogrithmus ist die Auswahl der genetischen Repräsentation.
    Nicht alle Arten von Evolutionären Algorithmen harmonieren mit jeder Repräsentation.

      \paragraph{Diskrete Repräsentation}
      \label{par:diskreteRepräsentation}
        Bei der diskreten Repräsentation wird das Genom als binären String dargestellt [0111110].
        Dieser binäre String kann anschliessend zu einem Phenotyp übersetzt werden.
        Zum Beispiel kann eine Bitsequenz, direkt zu einer Zahl als Phenotyp übersetzt werden [0011] -> 3.
        Ebenfalls kann eine diskrete Repäsentation eine Folge von beliebigen Zeichen annehmen [ABCDEF].
        D. Floreano und C. Mattiussi bringen dazu ein gutes Beispiel anhand des Travelling Sales Man Problemes an \cite[S.18]{book:bioInspired}:
        Jeder Buchstabe in der Sequenz repräsentiert dabei einen Ort, welchen es zu besuchen gilt.

      \paragraph{Reale Werte Repräsentation}
      \label{par:Reale Werte Repräsentation}
        Weiter kann die Reale Werte Repräsentation gewählt werden. Das Genom wird hierbei durch reele Zahlen repräsentiert.
        Beispielsweisse kann die optimale Position eines Zimmers (beste Flächennutzung) in einem Haus durch reale Zahlen dargestellt werden.

      \paragraph{Baum Repräsentationen}
      \label{par:Reale Werte Repräsentation}
        Das zu evolvierende Objekt kann durch einen Baum dargestellt werden.\\
        Baum Repräsenationen (siehe Abbidlung \ref{fig:baum}) werden vorallem bei der genetischen Programmierung eingesetzt.
        \begin{figure}[H]
          \Tree [.A [.B [.C eins ] [.D zwei ] ].B [.E {3 und 4} ] ].A
          \caption{Darstellung eines Baumes}
          \label{fig:baum}
        \end{figure}

    \subsubsection{Initiale Population}
      Die Grösse der Population hängt von den Eigenschaften des Suchraums und der Rechenkosten der Evaluation ab.
      Wichtig bei der Erstellung einer Population dabei ist, möglichst diverse Individuen zu erstellen,
      damit man sich nicht schon von Anfang an viele mögliche Lösung verspielt.

    \subsubsection{Fitness Funktion}
      Mit Hilfe der Fintess Funktion lassen sich Individuen beurteilen, wie gut geeiget Ihre Gene sind um die Problemstellung zu bewältigen.
      In der natürlichen Evolution ist die Fitness des Tieres, wie viele Nachkommen es erzeugen kann. In der technischen Welt jedoch muss der
      Anwenden sie jeweils selber definieren und nach seiner Problemstellung anpassen.

    \subsubsection{Selektionsoperation}
      Eine Selektionsoperation hilft einem möglichst gutgeignete Indivduen einer Generation zu selektieren.
      Die Selektierten bilden die Basis für die nächste Generation. Eine grosse Herausforderung beim Selektieren ist die Erhaltung der Diversität.

      \paragraph{Selektionsdruck}
        Als Selektionsdruck wird der Prozensatz der Individuen der aktuellen Generation, welche man verwendet um Nachkommen zu erzeugen, bezeichnet.
        Dabei werden am nur die fittesten Individuen selektiert. Zum Beispiel bei 10 Individuen werden nur die 4 selektiert, welche den höchsten Fitnesswert aufweisen.
        Selektion nach Selektionsdruck hat den Nachteil, dass die Diversität schnell verloren geht.

      \paragraph{Proportionle Selektion}
        Bei der proportionalen Selektion wählt man die Reproduktionsrate proportinal zur Fitness.
        Proportionale Selektion funktioniert schlecht wenn alle Individuen sehr ähnliche Fitnesswerte aufweisen oder es nur wenige/einen Ausreisser gibt.

      \paragraph{Rang-basierte Selektion}
        Es wird zuerst eine Rangliste erstellt und dann werden die Reproduktionswahrscheinlichkeiten proporional zum Rang zugeordnet.
        Diese Selektionsstrategy hat den Vorteil gegenüber der proportionalen Selektion, das es nicht darauf ankommt wie unterschiedlich die Individuen sind.

      \paragraph{Truncated Rang-basierte Selektion}
        Ähnlich wie die Rang-basierte Seleketion. Es werden jedoch nur die besten Individuen zur Reproduktion selektiert und
        jedes Individuum produziert die gleiche Anzahl Nachkommen.

      \paragraph{Turnier-basierte Selektion\label{par:Turnier}}

        Es werden \(k\) zufällig ausgewählte Individuen selektiert, das Individuum mit dem höchsten Fitnesswert erzeugt ein Nachkommen. Dies wird solange wiederholt bis man wieder so viele Nachkommen hat,
        wie die Ursprüngliche Generation Individuen hatte. Ein grosser Vorteil von Turniert-basierter Seleketion ist die gute Balanace zwischen Selektionsdruck und genetischer Diversität.

    \subsubsection{Rekombinationsfunktion}
        Bei der Rekombination werden jeweils Zwei Individuen selektiert und deren Genpaare werden rekombiniert (untereinander vertauscht).
        Rekombination ist nicht bei jeder Problemstellung hilfreich, es muss von Fall zu Fall neu bewertet werden ob sie eingesetzt wird.

        \paragraph{One-Point Crossover}
          Es wird ein zufälliger Cross-Over Punkt bestimmt an dem die Gene des Paares vertauscht werden.
          Anwendbar ist diese Strategie bei Diskreten und Reale Werte Repräsentationen.

        \paragraph{Multi-Point Crossover}
          Im Prinzip gleich wie One-Point Crossover, jedoch werden mehrere Punkte bestimmt und
          zwischen diesen Punkten werden dann die Gene ausgetauscht.

        \paragraph{Uniform Crossover}
          Die Gene werden an \(n\) zufälligen Stellen vertauscht, nur Anwendbar auf Reale Werte Repräsentationen.

        \paragraph{Arithmetic Crossiver}
          Der Durschnitt von den Genen an \(n\) zufälligen Positionen wird gebildet.
          Aus diesem Durchschnitt wird dann ein Nachkomme erzeugt.
          Ebenfalls nur anwendbar auf Reale Werte Repräsentationen.

        \paragraph{Sequenzen}
          Bei Sequenzen gilt es die Regel zu erhalten, dass alle Einträge nur einmal vorkommen dürfen. Wir erinnern uns an das Beispiel vom Travelling Sales Man \ref{par:diskreteRepräsentation}.
          Es wird sozusagen eine Unique Version von  Multi-Point Crossover angewendet.

        \paragraph{Bäume}
          Bei Bäumen wird ein zufälliger Teil des Baumes mit einem des anderen Nachkommens vertauscht.

    \subsubsection{Mutationsfunktion}
      Mutationen operieren direkt auf dem Individuum. Positonen eines Genomes werden mit einer bestimmten Wahrscheinlichkeit \(p_{m}\) mutiert.
      Es ist jedoch Vorsicht geboten, da gewisse Lösungen durch Mutationen verloren gehen können.

      \paragraph{Binäre Repräsentationen}
        Wenn die Repräsentation aus binären Werten besteht, werden die Bits geflippt (oder nur bestimmte Segmente).

      \paragraph{Reale Werte Repräsentationen}
        Es wird ein zufälliger Wert aus einer Gaus Verteilung \(N(0,\sigma)\) addiert.

      \paragraph{Sequenzen}
        Zwei zufällig ausgewählte Positionen werden miteinander vertauscht.

      \paragraph{Baum}
        Gewisse Teilabschnitte des Baumes werden mutiert.

    Genetische Representation: Erbgutdaten, genetic encoding \\
    Phenotyp: Aussehen des Individum -> Manifestation. In unserem Fall die 6-Beinigen Kreaturen als Pixelgrafik. \\
    Initialisieren -> Simulieren -> Selection -> Rekombination -> Mutation -> Analyse \\
    Simulieren bis Analayse solange wiederholen bis zufrieden mit Resultat

  \subsection{Arten von evolutionären Algorithmen}
  \label{sub:artenEvAlgos}
    Es gibt insgesammt 4 Arten von evolutionären Algorithmen:
    Genetische Algorithmen, Genetische Programmierung, Evolutinionäre Programmierung
    und Evolutionäre Strategien.

    \subsubsection{Genetische Algorithmen}
    \label{item:genAlgo}
      Die Genetische Algorithmen arbeiten mit binären Genom Repräsentationen.
      Es wird sowohl Rekombination als auch Mutation eingesetzt.

    \subsubsection{Genetische Programmierung}
    \label{item:genProg}
      Es werden Bäume als Repräsentation des Genoms eingesetzt. Wie bei den Genetischen Algorithmen,
      wird auch Rekombination und Mutation eingesetzt.

    \subsubsection{Evolutionäre Programmierung}
    \label{item:evProg}
      Bei der Evolutinären Programmierung wird ausschliesslich Mutation eingesetzt.
      Das Genom wird durch reale Werte repräsentiert. Als Selektionsoperator wird
      oft Turnier-basierte Selektion eingesetzt.

    \subsubsection{Evolutionäre Strategien}
    \label{item:evStrat}
        Ähnlich wie Evolutionäre Programmierung, aber die Varianz der Verteilung,
        welche für die Mutation gebraucht wird ist genetische codiert.
