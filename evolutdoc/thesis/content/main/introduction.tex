%
% Introduction
%

% !TEX root = ../../main.tex

\chapter{Einleitung}
\lipsum[33-37]
\section{Ausgangslage}
\lipsum[5-9]
Stand der Forschung 
\section{Zielsetzung}
Ziel ist es, in einer virtuellen Umwelt mit Hilfe evolutionärer Algorithmen künstliche Tiere entstehen zu lassen.
Evolviert werden die Geometrie und die Steuerung der Tiere, d.h. die Anordnung der Beine und deren Kontrolle.
Als Mass für die Fitness wird eine Funktion der Zeit, welche das Tier braucht,
um einen gegebenen Parcours zurückzulegen, verwendet.\\
Als Randbedingungen vorgegeben sind:
\begin{enumerate}
  \item Forderung einer physikalisch sinnvollen Bewegung
  \item Obere Grenze für die total aufgewandte Energie
  \item Beschränkung der abgegebenen Leistung
\end{enumerate}
Damit die Forderung nach einer physikalischen Bewegung erfüllt werden kann,
wird die Bewegung mit Hilfe einer Physik-Engine simuliert.
Diese Engine wird von den Studierenden ausgewählt und kann als Blackbox verwendet werden.
