%
% Introduction
%

% !TEX root = ../../main.tex

\chapter{Einleitung}

Evolutionäre Algorithmen helfen Problemstellung zu lösen, welche sonst nicht ohne so mathematisch lösbar sind.
Das heisst, dass der Problemraum zu gross für gängige mathematische Lösungsvorgehen ist.
Diese Arbeit hat sich das Ziel gesetzt, sechs Beinige Individuen und ihre Steuerung zu evolvieren.

\section{Ausgangslage}

Das notwendig theoretisch Wissen zur Bewältigung dieser Arbeit wurde mit Hilfe von den Buechern
Bio-inspired artificial intelligence: theories, methods, and technologies \cite[]{book:bioInspired} und
Evolutinäre Algorithmen \cite[]{book:evAlgo} erarbeitet.
  \todo[inline]{Nennt bestehende Arbeiten/Literatur zum Thema -> Literaturrecherche}
  \todo[inline]{Stand der Technik: Bisherige Lösungen des Problems und deren Grenzen}

\section{Zielsetzung}

  Als Mass für die Fitness wird eine Funktion der Zeit verwendet, welche das Tier braucht,
  um einen gegebenen Parcours zurückzulegen.
  Dabei müssen Individuen berücksichtigt werden, die den Parcours nicht vollständig ablaufen können.
  Die Fitnessfunktion berücksichtigt in diesem Fall die zurückgelegte Strecke.
  \\
  Als Randbedingungen vorgegeben sind:
  \begin{enumerate}
    \item Forderung einer physikalisch sinnvollen Bewegung
    \item Obere Grenze für die total aufgewandte Energie
    \item Beschränkung der abgegebenen Leistung
  \end{enumerate}
  Damit die Forderung nach einer physikalischen Bewegung erfüllt werden kann,
  wird die Bewegung mit Hilfe einer Physik-Engine simuliert.
  Diese Engine wird von den Studierenden ausgewählt und kann als Blackbox verwendet werden.

  \subsection{Requirements}

    \subsubsection{Fitness Funktion}

      Als Mass für die Fitness wird eine Funktion der Zeit, welche das Tier braucht,
      um einen gegebenen Parcours zurückzulegen, verwendet.
      Dabei müssen Individuen berücksichtigt werden,
      die den Parcours nicht vollständig ablaufen können.
      Die Fitnessfunktion berücksichtigt in diesem Fall die zurückgelegte Strecke.
      Wenn eine Sortierung nach dem höchsten Fitnesswert erfolgt,
      ist ein Individuum mit einer Funktion der Zeit immer höher zu rangieren,
      als ein Individuum welcher die zurückgelegte Streckte berücksichtigt.

      \paragraph{Vereinfachung}

        Es hat sich herausgestellt, dass eine Vereinfachung der Fitnessfunktion wenig Nachteile mit sich bringt.
        Darum wird nur die zurückgelegte Strecke der Tiere nach einer fixen Zeit als Fitnessfunktion berücksichtigt.
        Diese Vereinfachung wurde im Meeting vom 03.03.2016 mit dem Betreuer Prof\. Dr\. Rudolf Füchslin besprochen
        und als sinnvoll gewertet.

    \subsubsection{Parcours}

      Es muss mehr als nur einen Parcours für die Evolution der Individuen geben,
      ansonsten werden die Individuen nur auf den einen Parcours evolviert.
      Als Start-Parcours sollte ein möglichst einfach zu bewältigender Parcours dienen.
      Die Schwierigkeit des Parcours soll mit zunehmenden Generationen steigen und zufällig generiert werden.

    \subsubsection{Obergrenze für die totale aufgewandte Energie}

      Damit das Problem nicht trivial gelöst werden kann
      und ein Individuum einfach ins Unendliche beschleunigt werden kann,
      wird der Anspruch nach einer Obergrenze für die total aufgewandte Energie gestellt.
      Somit muss die abgebende Leistung beschränkt werden. Es werden Limiten eingeführt,
      um dieses Problem zu lösen.
