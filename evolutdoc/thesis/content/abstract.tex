%
% Abstract
%

% !TEX root = ../main.tex

\chapter{Abstract}

  Optimizing and solving problems in informatics and robotics are extensive tasks.
  Solving these problems requires expertise and costs time and resources.
  If a solution space grows too complex the formal solution methods reach a limit.
  Evolutionary algorithms produce through trial and error new approximative solutions.
  Following the principle of the surivival of the fittest these approximations become better over time.

  \smallskip

  In this paper we try to teach artificial animals with six legs how to move through a parcour.
  It is tested how the geometry and movement sequence can be evolved.
  Afterwards an analysis is done on how the movement sequence and the shape of an evolved individual looks.
  The hypotheses are examined based on the results of different simulations.
  An application was developped which implements the evolutionary algorithm.

  \smallskip

  During the simulation individuals of a population compete against each other on a parcour.
  Selection determines which individuals can reproduce themselves.
  Subsequently the genes of the offsprings are mutated.

  \smallskip

  Geometry and movement sequences are represented by an arbitrary long real-value representation.
  Thus all values can be diversified with ease.
  The results show that three different types of movement sequences were developped.
  A jump, row and roll movement.
  For each body shape 

  Geometrie und Bewegungsabläufe werden durch eine beliebig lange Reale-Werte-Repräsentation dargestellt.
  Dadurch können die Werte einfach variiert werden.
  Die Resultate zeigen, dass sich drei Typen von Bewegungsabläufen entwickelt haben.
  Entstanden sind Hüpf-, Ruder- und Rollbewegungen.
  Dabei zeigen sich je nach Bewegung verschiedene Tendenzen zu Körperformen.
  Rollende Individuen neigen zu kugelförmigen Körpern, Ruderer zu flachen Formen.
  Doch Hüpfer zeigen keinen eindeutigen Trend.
  Optimierungsbedarf besteht bei der Reaktion der Individuen auf Steigungen oder Gefälle im Parcours.
  \\
  Das Ziel der Arbeit mit evolutionären Algorithmen artifiziellen Tieren eine Fortbewegungsart beizugbringen,
  ist erreicht worden.
