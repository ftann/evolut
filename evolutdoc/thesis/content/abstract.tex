%
% Abstract
%

% !TEX root = ../main.tex

\chapter{Abstract}

  Optimizing and solving problems in informatcs and robotics are extensive tasks.
  Solving these problems requires expertise and costs time and resources.
  If a solution space grows too complex the formal solution methods reach a limit.
  Evolutionary algorithms produce through trial and error new approximative solutions.
  Following the principle of the surivival of the fittest these approximations become better over time.

  \smallskip

  In this paper we try to teach artificial animals how to walk.
  It is tested how the geometry and movement sequence can be evolved.

  %

  In dieser Arbeit wird versucht artifiziellen Tieren das Gehen beizubringen.
  Es wird untersucht, wie die Geometrie und Bewegungsablauf eines Tieres evolviert werden können.
  Anschliessend wird analysiert, wie der Bewegungsablauf sowie die Form eines evolvierten Tieres aussehen.
  Die Fragestellungen werden mit Ergebnissen verschiedener Simulationen untersucht.
  Dafür ist eine Applikation entwickelt worden, welche den evolutionären Algorithmus implementiert.

  \smallskip

  In der Simulation treten Individuen einer Population gegeneinander auf einem Parcours an.
  Durch Selektion wird festgelegt, welche Individuen sich reproduzieren können.
  Nachfolgend werden die Gene der Nachkommen mutiert.

  \smallskip

  Geometrie und Bewegungsablauf werden durch eine beliebig lange Reale-Werte-Repräsentation dargestellt.
  Dadurch können die Werte einfach variiert werden.
  Die Resultate zeigen, dass sich drei Typen von Bewegungsabläufen entwickelt haben.
  Entstanden sind Hüpf-, Ruder- und Rollbewegungen.
  Dabei zeigen sich je nach Bewegung verschiedene Tenzen zu Körperformen.
  Rollende Individuen neigen zu kugelförmigen Körpern, Ruderer zu flachen Formen.
  Doch Hüpfer zeigen keinen eindeutigen Trend.
  \\
  Optimierungsbedarf besteht bei der Reaktion der Individuen auf Steigung oder Gefälle im Parcours.
  \\
  Das Ziel der Arbeit mit evolutionären Algorithmen artifiziellen Tieren das Gehen beizugbringe ist erreicht worden.
