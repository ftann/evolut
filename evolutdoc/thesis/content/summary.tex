%
% Summary
%

% !TEX root = ../main.tex

\chapter{Zusammenfassung}

  In der Informatik und Robotik ist das Lösen und Optimieren von Problemstellungen eine aufwendige Aufgabe.
  Das Lösen dieser Problemstellungen erfordert Fachwissen, kostet Zeit und Ressourcen.
  \\
  Wird der Lösungsraum eines Problems zu komplex, stossen die formalen Lösungsmethoden an ihre Grenzen.
  Evolutionäre Algorithmen sind der natürlichen Evolution nachempfunden und
  produzieren durch ``trial and error'' neue Lösungen.
  Mit dem Prinzip ``surival of the fittest'' werden immer bessere Annäherungen produziert.

  \smallskip

  Diese Arbeit versucht sechsbeinigen künstlichen Tieren mit Hilfe eines evolutionären Algorithmus,
  das Fortbewegen durch einen Parcours beizubringen.
  Es wird untersucht, wie die Geometrie und Bewegungsablauf eines Tieres evolviert werden können.
  Anschliessend wird analysiert, wie der Bewegungsablauf sowie die Form eines evolvierten Tieres aussehen.
  Die Fragestellungen werden mit Ergebnissen verschiedener Simulationen untersucht.
  Dafür ist eine Applikation entwickelt worden, welche den evolutionären Algorithmus implementiert.

  \smallskip

  In der Simulation treten Individuen einer Population gegeneinander auf einem Parcours an.
  Durch die Selektion wird festgelegt, welche Individuen sich reproduzieren können.
  Nachfolgend werden die Gene der Nachkommen mutiert.

  \smallskip

  Geometrie und Bewegungsabläufe werden durch eine beliebig lange Reale-Werte-Repräsentation dargestellt.
  Dadurch können die Werte einfach variiert werden.
  Die Resultate zeigen, dass sich drei Typen von Bewegungsabläufen entwickelt haben.
  Entstanden sind Hüpf-, Ruder- und Rollbewegungen.
  Dabei zeigen sich je nach Bewegung verschiedene Tendenzen zu Körperformen.
  Rollende Individuen neigen zu kugelförmigen Körpern, Ruderer zu flachen Formen.
  Hingegen Hüpfer zeigen keinen eindeutigen Trend.
  Optimierungsbedarf besteht bei der Reaktion der Individuen auf Steigungen oder Gefälle im Parcours.
  \\
  Das Ziel der Arbeit mit evolutionären Algorithmen artifiziellen Tieren eine Fortbewegungsart beizugbringen,
  ist erreicht worden.
