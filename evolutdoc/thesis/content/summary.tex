%
% Summary
%

% !TEX root = ../main.tex

\chapter{Zusammenfassung}

  Das Optimieren vieler, praktisch relevanter und interessanter Probleme
  in der Informatik und Robotik sind aufwendige Aufgaben, die viel Rechenleistung benötigen.
  Wird der Lösungsraum eines Problems zu komplex, stossen formale Lösungsmethoden an ihre Grenzen.
  Evolutionäre Algorithmen sind der natürlichen Evolution nachempfunden und
  produzieren durch ``trial and error'' neue Lösungen.
  Mit dem Prinzip ``survival of the fittest'' werden immer bessere Annäherungen produziert.

  \medskip

  Diese Arbeit versucht sechsbeinigen künstlichen Tieren mit Hilfe eines evolutionären Algorithmus,
  das Fortbewegen durch einen Parcours beizubringen.
  Es wird untersucht, wie die Geometrie und Bewegungsablauf eines Tieres evolviert werden können.
  Anschliessend wird analysiert, wie der Bewegungsablauf sowie die Form eines evolvierten Tieres aussehen.
  Dabei werden folgende Forschungsfragen beantwortet:
  1. Wie kann eine Steuerung der Bewegung implementiert werden?
  2. Wie kann diese Steuerung evolviert werden?
  3. Wie kann die Geometrie der Tiere evolviert werden?
  4. Wie sehen der Bewegungsablauf und die Geometrie eines evolvierten Tieres aus?
  5. Liefern kleine Mutationswahrscheinlichkeiten bessere Fitnesswerte als grosse?
  Die Fragestellungen werden mit Ergebnissen verschiedener Simulationen untersucht.

  \medskip

  Die Resultate zeigen, dass sich drei Typen von Bewegungsabläufen entwickelt haben.
  Entstanden sind Hüpf-, Ruder- und Rollbewegungen.
  Dabei zeigen sich je nach Bewegung verschiedene Tendenzen zu Körperformen.
  Rollende Individuen neigen zu kugelförmigen Körpern, Ruderer zu flachen Formen.
  Hingegen Hüpfer zeigen keinen eindeutigen Trend.
  Optimierungsbedarf besteht bei der Reaktion der Individuen auf Steigungen oder Gefälle im Parcours.
